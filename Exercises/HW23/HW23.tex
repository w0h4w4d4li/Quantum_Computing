\documentclass{article}
\usepackage{amsmath}  % For advanced math typesetting
\usepackage{amsfonts} % For math fonts
\usepackage{amssymb}  % For additional symbols
\usepackage{graphicx} % For including images
\usepackage{geometry} % For page layout
\usepackage{fancyhdr} % For header and footer
\usepackage{setspace} % For line spacing
\usepackage{hyperref} % For hyperlinks
\usepackage{titlesec} % For customizing section titles
\usepackage{tikz} % For drawing
\usepackage{braket} % For braket notation

\geometry{a4paper, margin=1in}

% Header and Footer
\pagestyle{fancy}
\fancyhf{}
\fancyhead[L]{Chapter 7 - Solutions to Odd-Numbered Exercises}
\fancyhead[R]{\thepage}
\fancyfoot[C]{\thepage}

% Title Formatting
\titleformat{\section}{\Large\bfseries}{\thesection}{1em}{}
\titleformat{\subsection}{\large\bfseries}{\thesubsection}{1em}{}

% Title Page
\title{\textbf{Chapter 7} \\ \small Solutions to Odd-Numbered Exercises}
\author{
    MohamadAli Khajeian\footnote{khajeian@ut.ac.ir} \\ 
    \small \textit{Faculty of Engineering Sciences, University of Tehran, Iran} \\ 
}
\date{\today}

% Commands
\newcommand{\op}[2]{|#1\rangle \langle#2|}
\newcommand{\sand}[3]{\braket{#1 | #2 | #3}}
\newcommand{\sandop}[3]{\braket{#1 #2 #3}}
\newcommand{\tensor}[2]{#1 \otimes #2}

\begin{document}

\maketitle

\begin{abstract}
    This document presents the solution of "Quantum Computing Explained by David McMAHON" exercises.
\end{abstract}

\section*{Exercise 7.1}

The operator \(\vec{\sigma} \cdot \vec{n}\) is defined as
\[
\vec{\sigma} \cdot \vec{n} = \sigma_x n_x + \sigma_y n_y + \sigma_z n_z,
\]
and the unit vector \(\vec{n}\) is parameterized as
\[
\vec{n} = (\sin\theta\cos\phi, \sin\theta\sin\phi, \cos\theta).
\]
\begin{align*}
\vec{\sigma} \cdot \vec{n} &= \sigma_x\sin\theta\cos\phi + \sigma_y\sin\theta\sin\phi + \sigma_z\cos\theta \\
&= \big(\op{0}{1}+\op{1}{0}\big)\sin\theta\cos\phi + \big(-i\op{0}{1}+i\op{1}{0}\big)\sin\theta\sin\phi +\big(\op{0}{0}-\op{1}{1}\big)\cos\theta \\
&= \cos\theta\op{0}{0} + \sin\theta e^{-i\phi}\op{0}{1} + \sin\theta e^{i\phi}\op{1}{0} -\cos\theta \op{1}{1} = \begin{pmatrix} \cos\theta & \sin\theta e^{-i\phi} \\ \sin\theta e^{i\phi} & -\cos\theta \end{pmatrix}.
\end{align*}
The eigenvalue equation for the operator is
\[
(\vec{\sigma} \cdot \vec{n}) \ket{v} = \lambda \ket{v},
\]
where \(\lambda\) is the eigenvalue, and \(\ket{v} = \begin{pmatrix} a \\ b \end{pmatrix}\) is the eigenvector. To find the eigenvalues,
\begin{align*}
\det(\vec{\sigma} \cdot \vec{n} - \lambda I) = \det\bigg(&\big(\cos\theta-\lambda\big)\op{0}{0} + \sin\theta e^{-i\phi}\op{0}{1} \\&+ \sin\theta e^{i\phi}\op{1}{0} +  \big(-\cos\theta-\lambda\big)\theta \op{1}{1}\bigg)
\end{align*}
\begin{align*}
    \big(\cos\theta-\lambda\big)\big(-\cos\theta-\lambda\big) - \big(\sin\theta e^{-i\phi}\big)\big(\sin\theta e^{i\phi}\big)
    &= - \cos^2\theta + \lambda^2 - \sin^2\theta \\
    &= \lambda^2 - 1 = 0 \quad \implies \quad \lambda = \pm 1.
\end{align*}
to find eigenvector when \(\lambda = +1\)
\[
\begin{pmatrix} \cos\theta & \sin\theta e^{-i\phi} \\ \sin\theta e^{i\phi} & -\cos\theta \end{pmatrix}
\begin{pmatrix} a \\ b \end{pmatrix}
=
\begin{pmatrix} a \\ b \end{pmatrix}.
\]
this expands into the system of equations
\begin{align}
    \cos\theta a + \sin\theta e^{-i\phi} b = a \label{ev00}\\
    \sin\theta e^{i\phi} a - \cos\theta b = b. \label{ev01}   
\end{align}
from \ref{ev00} we have
\begin{align*}    
\frac{b}{a} &= \frac{1 - \cos\theta}{\sin\theta}e^{i\phi}\\
&= \frac{2 \sin^2 \frac{\theta}{2}}{2 \sin \frac{\theta}{2} \cos \frac{\theta}{2}} e^{i\phi} \\
&= \frac{\sin \frac{\theta}{2}}{\cos \frac{\theta}{2}} e^{i\phi}
\end{align*}
so
\begin{align*}
    \ket{+_n} = \cos \frac{\theta}{2} \ket{0} + e^{i\phi}\sin \frac{\theta}{2} \ket{1}
\end{align*}
to find eigenvector when \(\lambda = -1\)
\[
\begin{pmatrix} \cos\theta & \sin\theta e^{-i\phi} \\ \sin\theta e^{i\phi} & -\cos\theta \end{pmatrix}
\begin{pmatrix} a \\ b \end{pmatrix}
=
-\begin{pmatrix} a \\ b \end{pmatrix}.
\]
this expands into the system of equations
\begin{align}
    \cos\theta a + \sin\theta e^{-i\phi} b = -a \label{ev10}\\
    \sin\theta e^{i\phi} a - \cos\theta b = -b. \label{ev11}   
\end{align}
from \ref{ev11} we have
\begin{align*}    
\frac{a}{b} &= \frac{\cos\theta - 1}{\sin\theta}e^{-i\phi}\\
&= \frac{-2 \sin^2 \frac{\theta}{2}}{2 \sin \frac{\theta}{2} \cos \frac{\theta}{2}} e^{-i\phi} \\
&= \frac{-\sin \frac{\theta}{2}}{\cos \frac{\theta}{2}} e^{-i\phi}
\end{align*}
so
\begin{align*}
    \ket{-_n} = \sin \frac{\theta}{2} \ket{0} - e^{i\phi}\cos \frac{\theta}{2} \ket{1}
\end{align*}
\section*{Exercise 7.3}
we know
\begin{align*}
   \ket{\beta_{00}} = \frac{\ket{00}+\ket{11}}{\sqrt{2}}, \quad  \ket{\beta_{01}} = \frac{\ket{01}+\ket{10}}{\sqrt{2}}
\end{align*}
for $\beta_{00}$ and $\beta_{01}$ 
\begin{align*}
    \tensor{\text{Z}}{\text{Z}}\ket{\beta_{00}} = (-1)^0\frac{\ket{00}+\ket{11}}{\sqrt{2}} = (-1)^{y}\ket{\beta_{00}}, \quad \tensor{\text{Z}}{\text{Z}}\ket{\beta_{01}} = (-1)^1\frac{\ket{01}+\ket{10}}{\sqrt{2}} = (-1)^{y}\ket{\beta_{01}}
\end{align*}
\section*{Exercise 7.5}
we know
\begin{align*}
    \ket{\beta_{xy}} = \frac{\ket{0y}+(-1)^{x}\ket{1\bar{y}}}{\sqrt{2}}
\end{align*}
\begin{align*}
    \tensor{\text{Y}}{\text{Y}}\ket{\beta_{xy}} =  \tensor{\text{Y}}{\text{Y}}\bigg(\frac{\ket{0y}+(-1)^{x}\ket{1\bar{y}}}{\sqrt{2}}\bigg) &= \frac{(i)((-1)^{y}i)\ket{1\bar{y}}+(-1)^{x}(-i)((-1)^{\bar{y}}i)\ket{0y}}{\sqrt{2}} \\
    &= \frac{(-1)(-1)^{y}\ket{1\bar{y}}+(-1)^{x}(-1)^{\bar{y}}\ket{0y}}{\sqrt{2}} \\
    &= \frac{(-1)(-1)^{y}(-1)^{x}(-1)^{x}\ket{1\bar{y}}+(-1)^{x}(-1)^{\bar{y}}\ket{0y}}{\sqrt{2}} \\
    &= \frac{(-1)^{x+\bar{y}}(-1)^{x}\ket{1\bar{y}}+(-1)^{x+\bar{y}}\ket{0y}}{\sqrt{2}} = (-1)^{x+\bar{y}}\ket{\beta_{xy}}
\end{align*}
\section*{Exercise 7.7}
\section*{Exercise 7.9}
\begin{align*}
    \rho = \begin{pmatrix}
        \sin^2\theta & e^{-i\phi}\sin\theta\cos\theta \\
        e^{i\phi}\sin\theta\cos\theta & \cos^2\theta
    \end{pmatrix} &= \sin^2\theta\op{0}{0} + e^{-i\phi}\sin\theta\cos\theta\op{0}{1} + \\&e^{i\phi}\sin\theta\cos\theta\op{1}{0} + \cos^2\theta\op{1}{1}
\end{align*}
need to get $c_{1}$, $c_{2}$ and $c_{3}$
\begin{align*}
    c_{0} = \braket{\sigma_0} = \text{Tr}(\rho\hspace{0.1cm} \sigma_0)\\
    c_{1} = \braket{\sigma_1} = \text{Tr}(\rho\hspace{0.1cm} \sigma_1)\\
    c_{2} = \braket{\sigma_2} = \text{Tr}(\rho\hspace{0.1cm} \sigma_2)\\
    c_{3} = \braket{\sigma_3} = \text{Tr}(\rho\hspace{0.1cm} \sigma_3)
\end{align*}
then
\begin{align*}
    c_{0} &= 1, \\
    c_{1} &= \text{Tr}\bigg(e^{i\phi}\sin\theta\cos\theta\op{1}{1}+...+e^{-i\phi}\sin\theta\cos\theta\op{0}{0}\bigg)\\
    &= e^{i\phi}\sin\theta\cos\theta + e^{-i\phi}\sin\theta\cos\theta = \bigg(e^{i\phi}+e^{-i\phi}\bigg)\sin\theta\cos\theta = \bigg(2\cos\phi\bigg)\frac{\sin2\theta}{2} = \cos\phi\sin2\theta,\\
    c_{2} &= \text{Tr}\bigg(-ie^{i\phi}\sin\theta\cos\theta\op{1}{1}+...+ie^{-i\phi}\sin\theta\cos\theta\op{0}{0}\bigg)\\
    &= -ie^{i\phi}\sin\theta\cos\theta + ie^{-i\phi}\sin\theta\cos\theta = -i\bigg(e^{i\phi}-e^{-i\phi}\bigg)\sin\theta\cos\theta = -i\bigg(2i\sin\phi\bigg)\frac{\sin2\theta}{2} = \sin\phi\sin2\theta, \\
    c_{3} &= \text{Tr}\bigg(\sin^2\theta\op{0}{0}+...-\cos^2\theta\op{1}{1}\bigg)\\
    &= \sin^2\theta - \cos^2\theta = -\cos2\theta,
\end{align*}
so we have $\displaystyle\rho = \frac{1}{2}\bigg(\sum_i c_i \sigma_i\bigg)$.
\section*{Exercise 7.11}
\begin{align}
    \op{\beta_{00}}{\beta_{00}} = \frac{1}{4}\bigg(\tensor{\text{I}}{\text{I}}+\tensor{\text{X}}{\text{X}}-\tensor{\text{Y}}{\text{Y}}+\tensor{\text{Z}}{\text{Z}}\bigg) \label{7.39}
\end{align}
to proof \ref{7.39} we have
 \begin{align*}
    &\frac{1}{4}\bigg(\bigg(\tensor{\big(\op{0}{0}+\op{1}{1}\big)}{\big(\op{0}{0}+\op{1}{1}\big)}\bigg)+\bigg(\tensor{\big(\op{0}{1}+\op{1}{0}\big)}{\big(\op{0}{1}+\op{1}{0}\big)}\bigg)\\&
    -\bigg(\tensor{\big(-i\op{0}{1}+i\op{1}{0}\big)}{\big(-i\op{0}{1}+i\op{1}{0}\big)}\bigg)+\bigg(\tensor{\big(\op{0}{0}-\op{1}{1}\big)}{\big(\op{0}{0}-\op{1}{1}\big)}\bigg)\bigg) \\
    &= \frac{1}{4}\bigg(\big(\op{00}{00}+\op{01}{01}+\op{10}{10}+\op{11}{11}\big)+\big(\op{00}{11}+\op{01}{10}+\op{10}{01}+\op{11}{00}\big)\\
    &-\big(-\op{00}{11}+\op{01}{10}+\op{10}{01}-\op{11}{00}\big)+\big(\op{00}{00}-\op{01}{01}-\op{10}{10}+\op{11}{11}\big)\bigg)\\
    &=\frac{1}{4}\bigg(2\op{00}{00}+2\op{00}{11}+2\op{11}{00}+2\op{11}{11}\bigg)\\
    &=\frac{1}{2}\bigg(\op{00}{00}+\op{00}{11}+\op{11}{00}+\op{11}{11}\bigg) = \bigg(\frac{\ket{00}+\ket{11}}{\sqrt{2}}\bigg)\bigg(\frac{\bra{00}+\bra{11}}{\sqrt{2}}\bigg) = \op{\beta_{00}}{\beta_{00}}
 \end{align*}
 \section*{Exercise 7.13}
\begin{align*}
    \ket{\psi} = \bigg(\frac{\ket{0}-\ket{1}}{\sqrt{2}}\bigg)\otimes\bigg(\frac{\ket{0}-\ket{1}}{\sqrt{2}}\bigg) = \frac{1}{2}\bigg(\ket{00}-\ket{01}-\ket{10}+\ket{11}\bigg)
\end{align*}
we can get density matrix
\begin{align*}
    \rho &= \op{\psi}{\psi} = \frac{1}{2}\bigg(\ket{00}-\ket{01}-\ket{10}+\ket{11}\bigg)\frac{1}{2}\bigg(\bra{00}-\bra{01}-\bra{10}+\bra{11}\bigg) \\
    &= \frac{1}{4}\bigg(\op{00}{00}-\op{00}{01}-\op{00}{10}-\op{00}{11}
    -\op{01}{00}+\op{01}{01}+\op{01}{10}-\op{01}{11} \\
    &-\op{10}{00}+\op{10}{01}+\op{10}{10}-\op{10}{11}+\op{11}{00}-\op{11}{01}-\op{11}{10}+\op{11}{11}\bigg)
\end{align*}
need to get $c_{11}$, $c_{22}$ and $c_{33}$
\begin{align*}
    c_{11} = \braket{\tensor{\sigma_1}{\sigma_1}} = \text{Tr}(\rho\hspace{0.1cm} \tensor{\sigma_1}{\sigma_1})\\
    c_{22} = \braket{\tensor{\sigma_2}{\sigma_2}} = \text{Tr}(\rho\hspace{0.1cm} \tensor{\sigma_2}{\sigma_2})\\
    c_{33} = \braket{\tensor{\sigma_3}{\sigma_3}} = \text{Tr}(\rho\hspace{0.1cm} \tensor{\sigma_3}{\sigma_3})
\end{align*}
to get $c_{11}$
\begin{align*}
    \tensor{\sigma_1}{\sigma_1} = \big(\op{00}{11}+\op{01}{10}+\op{10}{01}+\op{11}{00}\big)
\end{align*}
then
\begin{align*}
    c_{11} &= \frac{1}{4}\text{Tr}\bigg(\op{11}{11}+...+\op{10}{10}+...+\op{01}{01}+...-\op{00}{00}\bigg)\\
    &= \frac{1}{2}
\end{align*}
to get $c_{22}$
\begin{align*}
    \tensor{\sigma_2}{\sigma_2} = \big(-\op{00}{11}+\op{01}{10}+\op{10}{01}-\op{11}{00}\big)
\end{align*}
then
\begin{align*}
    c_{22} &= \frac{1}{4}\text{Tr}\bigg(-\op{11}{11}+...+\op{10}{10}+...+\op{01}{01}+...+\op{00}{00}\bigg)\\
    &= \frac{1}{2}
\end{align*}
to get $c_{33}$
\begin{align*}
    \tensor{\sigma_3}{\sigma_3} = \big(\op{00}{00}-\op{01}{01}-\op{10}{10}+\op{11}{11}\big)
\end{align*}
then
\begin{align*}
    c_{33} &= \frac{1}{4}\text{Tr}\bigg(\op{00}{00}+...-\op{01}{01}+...-\op{10}{10}+...+\op{11}{11}\bigg)\\
    &= 0
\end{align*}
so we have
\begin{align*}
    |c_{11}|+|c_{22}|+|c_{33}| = \frac{1}{2}+\frac{1}{2}+0 \leq 1
\end{align*}
so $\psi$ is a product state.
\end{document}
