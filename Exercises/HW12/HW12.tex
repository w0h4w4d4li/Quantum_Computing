\documentclass[12pt]{article}
\usepackage{amsmath}  % For advanced math typesetting
\usepackage{amsfonts} % For math fonts
\usepackage{amssymb}  % For additional symbols
\usepackage{graphicx} % For including images
\usepackage{geometry} % For page layout
\usepackage{fancyhdr} % For header and footer
\usepackage{setspace} % For line spacing
\usepackage{hyperref} % For hyperlinks
\usepackage{titlesec} % For customizing section titles
\usepackage{tikz} % For drawing
\usepackage{braket} % For braket notation

\geometry{a4paper, margin=1in}

% Header and Footer
\pagestyle{fancy}
\fancyhf{}
\fancyhead[L]{Chapter 4 - Solutions to Even-Numbered Exercises}
\fancyhead[R]{\thepage}
\fancyfoot[C]{\thepage}

% Title Formatting
\titleformat{\section}{\Large\bfseries}{\thesection}{1em}{}
\titleformat{\subsection}{\large\bfseries}{\thesubsection}{1em}{}

% Title Page
\title{\textbf{Chapter 4} \\ \small Solutions to Even-Numbered Exercises}
\author{
    MohamadAli Khajeian\footnote{khajeian@ut.ac.ir} \\ 
    \small \textit{Faculty of Engineering Sciences, University of Tehran, Iran} \\ 
}
\date{\today}

\begin{document}

\maketitle

\begin{abstract}
    This document presents the solution of "Quantum Computing Explained by David McMAHON" exercises.
\end{abstract}

\section*{Exercise 4.2}
The basis states for \( H \equiv \mathbb{C}^4 \) can be constructed by using \( |+\rangle, |-\rangle \) as the basis for \( H_1 \) and \( H_2 \).
\begin{equation*}
|w_1\rangle = |+\rangle|+\rangle
\end{equation*}
\begin{equation*}
|w_2\rangle = |+\rangle|-\rangle
\end{equation*}
\begin{equation*}
|w_3\rangle = |-\rangle|+\rangle 
\end{equation*}
\begin{equation*}
|w_4\rangle = |-\rangle|-\rangle
\end{equation*}
we have
\[
\langle w_3 | w_4 \rangle = (\langle -| \langle +|)(|- \rangle |- \rangle) = \langle -|- \rangle \langle +|- \rangle = (1)(0) = 0
\]
\[
\langle w_4 | w_3 \rangle = (\langle -| \langle -|)(|- \rangle |+\rangle) = \langle -|- \rangle \langle -|+\rangle = (1)(0) = 0
\]
\section*{Exercise 4.4}
To calculate the tensor product of
\[
|\psi\rangle = \frac{1}{\sqrt{2}} \begin{pmatrix} 1 \\ 1 \end{pmatrix} \quad \text{and} \quad |\phi\rangle = \frac{1}{2} \begin{pmatrix} 1 \\ \sqrt{3} \end{pmatrix}
\]
we have
\[
|\psi\rangle \otimes |\phi\rangle = \frac{1}{\sqrt{2}} \begin{pmatrix} 1 \\ 1 \end{pmatrix} \otimes \frac{1}{2} \begin{pmatrix} 1 \\ \sqrt{3} \end{pmatrix} = \frac{1}{2\sqrt{2}} \begin{pmatrix} 1 \\ 1 \end{pmatrix} \otimes \begin{pmatrix} 1 \\ \sqrt{3} \end{pmatrix}
\]
then
\[
|\psi\rangle \otimes |\phi\rangle = \frac{1}{2\sqrt{2}} \begin{pmatrix} 1 \\ \sqrt{3} \\ 1 \\ \sqrt{3} \end{pmatrix}.
\]
\section*{Exercise 4.6}
No we can't.
\section*{Exercise 4.8}
To show that \((A \otimes B)^\dagger = A^\dagger \otimes B^\dagger\), assume
\begin{equation}
    \label{e1}
    \ket{x} = \ket{x_1} \otimes \ket{x_2}
\end{equation}
\begin{equation}
    \label{e2}
    \ket{y} = \ket{y_1} \otimes \ket{y_2}
\end{equation}
when you apply the operator \(A \otimes B\) to a product state, say \(|i\rangle \otimes |j\rangle\), it acts as follows
\[
(A \otimes B)(|i\rangle \otimes |j\rangle) = A|i\rangle \otimes B|j\rangle.
\]
now using \ref{e1} and \ref{e2},
\begin{equation}
\label{e3}
\langle y | (A \otimes B) | x \rangle = \langle y_1 | A | x_1 \rangle \cdot \langle y_2 | B | x_2 \rangle.    
\end{equation}
we know
\[
\langle x | (A \otimes B)^\dagger | y \rangle = \langle y | (A \otimes B) | x \rangle^*.
\]
using \ref{e3},
\begin{align*}
\langle x | (A \otimes B)^\dagger | y \rangle &= \langle y | (A \otimes B) | x \rangle^* \\
&= \langle y_1 | A | x_1 \rangle^* \cdot \langle y_2 | B | x_2 \rangle^* \\  
&= \langle x_1 | A^\dagger | y_1 \rangle \cdot \langle x_2 | B^\dagger | y_2 \rangle.
\end{align*}
we have
\[
\langle x | (A^\dagger \otimes B^\dagger) | y \rangle = \langle x_1 | A^\dagger | y_1 \rangle \cdot \langle x_2 | B^\dagger | y_2 \rangle = \langle x | (A \otimes B)^\dagger | y \rangle.
\]
since both expressions yield the same result, we conclude that
\[
(A \otimes B)^\dagger = A^\dagger \otimes B^\dagger.
\]
\section*{Exercise 4.10}
Let's write down the Pauli matrices
\[
X = \begin{pmatrix} 0 & 1 \\ 1 & 0 \end{pmatrix}, \quad Y = \begin{pmatrix} 0 & -i \\ i & 0 \end{pmatrix}
\]
now we have
\[
X \otimes Y = 
\begin{pmatrix}
(0)Y & (1)Y \\
(1)Y & (0)Y
\end{pmatrix}
=
\begin{pmatrix}
0 & 0 & 0 & -i \\
0 & 0 & i & 0 \\
0 & -i & 0 & 0 \\
i & 0 & 0 & 0
\end{pmatrix}.
\]
\end{document}
